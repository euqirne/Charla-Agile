\section{Créditos}

 \begin{frame}
  \frametitle{About Transparencias}
  \transdissolve<1->
  \only<+> {
  Estas transparencias son libres. \\
  Siénte libre de usarla si quieres.\\
  Basta mostrar a sus autores:

  \begin{itemize}
   \item Miguel Ángel García Martínez
   \item David Villa Alises
  \end{itemize}
  }
  \only<+>{
  Las transparencias están aquí:

  \begin{center}
   https://github.com/magmax/Charla-Agile\\
  \end{center}
  }
  \only<+->{
  Para realizar las transparencias se ha usado:

  \begin{itemize}
   \item<+-> \textbf{Emacs}, para editar los textos.
   \item<+-> \textbf{wget}, para descargar las imágenes.
   \item<+-> \textbf{Make}, para automatizar los procesos.
   \item<+-> \textbf{\LaTeX}, para la composición.
   \item<+-> \textbf{Beamer}, que es una librería \LaTeX \ para presentaciones.
   \item<+-> \textbf{rubber}, para generar el documento.
   \item<+-> \textbf{convert}, para transformar imágenes.
   \item<+-> \textbf{git}, para gestionar versiones.
  \end{itemize}
  }
  \only<+->{
  Todas son herramientas libres.
  }
 \end{frame}


 \begin{frame}
  \frametitle{About Agile-CR}
  \transdissolve<1->
  \only<+>{
  Actividades Actuales:
  \begin{itemize}
   \item Reuniones cada dos semanas
   \item La próxima reunión es el miércoles que viene (23 de marzo)
   \item ¡¡Os esperamos!!
  \end{itemize}
  }
  \only<+>{
  Actividades Futuras:
  \begin{itemize}
   \item  Club de lectura.
   \item  CodeRetreat.
   \item  DevOpen Space.
   \item  Lo que se nos ocurra.
  \end{itemize}
  }
  \only<+> {
  Agile-CR es parte de Agile-Spain
  \begin{itemize}
   \item http://groups.google.com/group/agile-cr
   \item http://groups.google.com/group/agile-spain
  \end{itemize}
  }
 \end{frame}

 \begin{frame}
  \frametitle{About Miguel Ángel}
  \transdissolve<1->
  Podéis contactar conmigo si queréis:

  \begin{itemize}
   \item<+-> miguelangel@magmax.org
   \item<+-> http://www.magmax.org
   \item<+-> http://twitter.com/magmax9
  \end{itemize}
 \end{frame}
