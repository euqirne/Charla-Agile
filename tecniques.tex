 \section{Técnicas}

  \subsection{Palabrotas}

  \begin{frame}
   \frametitle{Scrum, Kanban, Scrumban}
   \framesubtitle{Qué son}

   \begin{itemize}
    \item Son marcos de trabajo (\textit{frameworks}).
    \item 4 tonterías que funcionan.
    \item Evitan la mentira.
   \end{itemize}
  \end{frame}

  \begin{frame}
   \frametitle{Scrum, Kanban, Scrumban}
   \framesubtitle{Scrum en 6 palabras}

   \begin{itemize}
    \item<1-> División en \textbf{Sprints}.
    \item<2-> Reunión de \textbf{planificación} de Sprint.
    \item<3-> \textbf{Scrum} diario.
    \item<4-> Reunión de \textbf{seguimiento} de Sprint opcional.
    \item<5-> Reunión de \textbf{evaluación} de Sprint.
    \item<6-> \textbf{Panel}: Pendiente, Asignado, Terminado. Burndown.
   \end{itemize}
  \end{frame}

  \begin{frame}
   \frametitle{Scrum, Kanban, Scrumban}
   \framesubtitle{Test Driven Development}
   \transdissolve<1>
   \begin{center}
    {\small (Foto realizada por Luismi Caballé. ¡Gracias!)}
    \includegraphics[width=8cm]{images/panelScrum.jpg}
   \end{center}
  \end{frame}

  \begin{frame}
   \frametitle{Scrum, Kanban, Scrumban}
   \framesubtitle{Aplicaciones: Kunagi}
   \transdissolve<1->
   \begin{center}
    \only<1>{
    \includegraphics[width=11cm]{images/kunagiwhiteboard.png}
    }
    \only<2>{
    \includegraphics[width=11cm]{images/kunagidashboard.png}
    }
   \end{center}

  \end{frame}

  \begin{frame}
   \frametitle{Scrum, Kanban, Scrumban}
   \framesubtitle{Aplicaciones: Agilefant}
   \transdissolve<1->
   \begin{center}
    \only<1>{
    \includegraphics[width=7cm]{images/agilefantstories.png}
    }
    \only<2>{
    \includegraphics[width=7cm]{images/agilefantburndown.png}
    }
   \end{center}

  \end{frame}

  \begin{frame}
   \frametitle{Pomodoro}
   \framesubtitle{Agilefant}

   \begin{columns}
    \begin{column}{4cm}
     \includegraphics[width=4cm]{images/pomodoro.png}
    \end{column}
    \begin{column}{7cm}
     \begin{itemize}
      \item<1-> 25 minutos de máxima concentración.
      \item<2-> 5 minutos descanso.
      \item<3-> 2 listas.
      \item<4-> Marcas.
     \end{itemize}
    \end{column}
   \end{columns}
  \end{frame}


  \subsection{Herramientas}

  \begin{frame}
   \frametitle{Sistemas de control de versiones}
   \framesubtitle{Para todos los gustos}

   \begin{itemize}
    \item CVS
    \item Subversion
    \item Bazaar
    \item Mercurial
    \item Git
   \end{itemize}
  \end{frame}

  \begin{frame}
   \frametitle{Sistemas de control de versiones}
   \framesubtitle{Órdenes}
   \transdissolve<1->

   \only<+>{
   Subversion
   \begin{itemize}
    \item Crear repo: \$ svnadmin create
    \item Descargar: \$ svn checkout
    \item Añadir: \$ svn add
    \item Actualizar: \$ svn update
    \item Subir: \$ svn commit
   \end{itemize}
   }
   \only<+>{
   Mercurial
   \begin{itemize}
    \item Crear repo: \$ hg init
    \item Descargar: \$ hg clone
    \item Añadir: \$ hg add
    \item Actualizar: \$ hg update
    \item Subir: \$ hg commit
   \end{itemize}
   }
   \only<+>{
   Git
   \begin{itemize}
    \item Crear repo: \$ git init
    \item Descargar: \$ git clone
    \item Añadir: \$ git add
    \item Actualizar: \$ git update
    \item Subir: \$ git commit
   \end{itemize}
   }
   \only<+>{
   Bazaar
   \begin{itemize}
    \item Crear repo: \$ bzr init
    \item Descargar: \$ bzr branch
    \item Añadir: \$ bzr add
    \item Actualizar: \$ bzr merge
    \item Subir: \$ bzr commit
   \end{itemize}
  }
  \end{frame}

  \begin{frame}
   \frametitle{Sistemas de control de versiones}
   \framesubtitle{¡¡Gratis!!}

   \begin{itemize}
    \item <+-> https://github.com (Git)
    \item <+-> https://bitbucket.org (Mercurial)
    \item <+-> http://code.google.com/hosting (SVN/Mercurial)
    \item <+-> http://alioth.debian.org (todos; Debian)
    \item <+-> http://savannah.gnu.org (todos)
    \item <+-> http://gna.org (todos)
   \end{itemize}
  \end{frame}

  \begin{frame}
   \frametitle{Integración contínua}
   \framesubtitle{Concepto}

   \begin{itemize}
    \item<+-> Propuesta por Martin Fowler.
    \item<+-> Consiste en hacer integraciones automáticas contínuamente.
    \item<+-> Permite detectar problemas mucho antes.
   \end{itemize}
  \end{frame}


  \begin{frame}
   \frametitle{Integración contínua}
   \framesubtitle{Proceso}

   Cada cierto tiempo:

   \begin{itemize}
    \item<+-> Se descargan los fuentes de un gestor de versiones.
    \item<+-> Se compila.
    \item<+-> Se ejecutan los tests.
    \item<+-> Se crean informes.
   \end{itemize}
  \end{frame}

  \begin{frame}
   \frametitle{Integración contínua}
   \framesubtitle{Aplicaciones}

   \begin{itemize}
    \item Continuum (Apache)
    \item Jenkins (CC)
    \item CruiseControl (BSD)
   \end{itemize}
  \end{frame}


  \begin{frame}
   \frametitle{Integración contínua}
   \framesubtitle{Continuum}

   \begin{center}
    \includegraphics[width=11cm]{images/continuum.png}
   \end{center}
  \end{frame}

  \begin{frame}
   \frametitle{Integración contínua}
   \framesubtitle{Jenkins}

   \begin{center}
    \includegraphics[width=16cm]{images/jenkins.png}
   \end{center}
  \end{frame}

  \begin{frame}
   \frametitle{Integración contínua}
   \framesubtitle{CruiseControl}

   \begin{center}
    \includegraphics[width=11cm]{images/cruisecontrol.png}
   \end{center}
  \end{frame}



  \subsection{TDD}

  \begin{frame}
   \frametitle{Mediciones}
   \framesubtitle{¿Qué es el código limpio?}

   \begin{center}
    \includegraphics[width=7cm]{images/wtfm.jpg}
   \end{center}
  \end{frame}



  \begin{frame}
   \frametitle{Test Driven Development}
   \framesubtitle{Qué significa}

   Pasos:
   \begin{enumerate}
    \item<+-> Escribir el \textbf{test}.
    \item<+-> Escribir el \textbf{código mínimo} que pasa el test.
    \item<+-> \textbf{Refactorizar}.
   \end{enumerate}
  \end{frame}


  \begin{frame}
   \frametitle{Código limpio}
   \framesubtitle{Importancia}
   \transdissolve<1->
   \begin{columns}
    \begin{column}{4cm}
     \includegraphics[width=4cm]{images/cleancode.png}
    \end{column}
    \begin{column}{7cm}
     \begin{itemize}
      \item<+-> El código se escribe 1 vez, pero se lee muchas.
      \item<+-> Las pruebas pueden ser la mejor documentación.
      \item<+-> Tanto las pruebas como el código nos cuentan una historia.
     \end{itemize}
    \end{column}
   \end{columns}
  \end{frame}


  \begin{frame}
   \frametitle{Código limpio}
   \framesubtitle{Importancia de las pruebas}
   \transdissolve<1->
   Escribir pruebas \ldots
   \begin{itemize}
    \item<+-> \ldots nos ayudará a escribir mejores pruebas.
    \item<+-> \ldots nos obligará a escribir código fácil de probar.
    \item<+-> \ldots mejorará la comprensión y usabilidad de nuestro código.
   \end{itemize}
  \end{frame}

  \begin{frame}
   \begin{center}
    \includegraphics[width=10cm]{images/profesional.jpg}
   \end{center}
  \end{frame}

 \begin{frame}
  \frametitle{Profesionales}
  \framesubtitle{Ward Cunningham}
  \transdissolve<1->
  \begin{columns}
   \begin{column}{6cm}
    \only<1>{
    \begin{itemize}
     \item Creador del ``\textit{wiki}''.
     \item Representante de XP.
     \item Trabajó para Microsoft en el ``grupo prácticas y patrones''
     \item Director del ``Committer Community Development'' de la
	   Fundación Eclipse.
    \end{itemize}
    }
    \only<2>{
    ``You know you are working on
    clean code when each routine
    you read turns out to be pretty
    much what you expected.''
    }
    \only<3>{
    ``You can call it beautiful code
    when the code also makes it
    look like the language was
    made for the problem''
    }
   \end{column}
   \begin{column}{5cm}
    \includegraphics[width=5cm]{images/mentorsCunningham.jpg}
   \end{column}
  \end{columns}
 \end{frame}


 \begin{frame}
  \frametitle{Profesionales}
  \framesubtitle{Kent Beck}
  \transdissolve<1->
  \begin{columns}
   \begin{column}{4cm}
    \includegraphics[width=4cm]{images/mentorsKentBeck.jpg}
   \end{column}
   \begin{column}{6cm}
    \only<1>{
    \begin{itemize}
     \item Creador de XP.
     \item Creador de metodologías TDD.
     \item Creador de JUnit con Erich Gamma.
    \end{itemize}
    }
    \only<2>{
    Rules for simple design:\\
    1. Runs all the tests\\
    2. Contains no duplication\\
    3. Expresses all the intent of the programmer\\
    4. Minimizes the number of classes and methods.\\
    The rules are given in order of importance.
    }
   \end{column}
  \end{columns}
 \end{frame}

  {
   \usebackgroundtemplate{\includegraphics[width=\paperwidth]{images/ventanaRota.jpg}}

  \begin{frame}
   \frametitle{Código limpio}
   \framesubtitle{Teoría de la ventana rota}
   .%sí, esto es cutre, pero no importa demasiado.
   \\[6pc]
   El caos comienza por una ventana rota.

  \end{frame}
  }

  {
   \usebackgroundtemplate{}

  \begin{frame}
   \frametitle{Código limpio}
   \framesubtitle{Deuda técnica}

   \begin{columns}
    \begin{column}{4cm}
     \includegraphics[width=4cm]{images/domino.jpg}
     \end{column}
    \begin{column}{7cm}
     La ``\textbf{Deuda técnica}'' es como saltarse una pieza:
     \begin{itemize}
      \item<+-> Iremos más rápido.
      \item<+-> El programa funcionará.
      \item<+-> Tarde o temprano ocurrirá un problema y tendremos que
	    averiguar por qué ha ocurrido.
     \end{itemize}
    \end{column}
   \end{columns}
  \end{frame}
  }
