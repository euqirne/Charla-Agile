 \section{Motivación}

  \subsection{Malas prácticas}

  \begin{frame}
   \frametitle{Programación tradicional}
   \framesubtitle{Lo que siempre nos han enseñado}

   \transboxin<1>

   \begin{itemize}
    \item<1-> Especificar \textbf{los requisitos}.
    \item<2-> \textbf{Aceptar} los requisitos.
    \item<3-> \textbf{Planificación} de tiempos.
    \item<4-> \textbf{Desarrollar} la aplicación.
    \item<5-> \textbf{Probar} que todo funciona.
    \item<6-> \textbf{Documentar} lo que se ha hecho.
    \item<7-> \textbf{Entregar} al cliente.
   \end{itemize}
  \end{frame}

  \begin{frame}
   \frametitle{Programación tradicional}
   \framesubtitle{Lo que realmente querían decir}

   \transboxout<1>

   \begin{itemize}
    \item<1-> Hacer la carta a \textbf{los Reyes Magos}.
    \item<2-> \textbf{Dividir} el tiempo que digan los programadores entre 10.
    \item<3-> Dejar que los programadores \textbf{se diviertan}.
    \item<4-> \textbf{Fustigar} a los programadores por no tenerlo a tiempo.
    \item<5-> \textbf{Transmitir} a los programadores las quejas del cliente.
   \end{itemize}
  \end{frame}

   \begin{frame}
    \frametitle{Programación tradicional}
    \framesubtitle{Causas}

    \transdissolve<1>

    \begin{itemize}
     \item<1-> Carta de los Reyes Magos != Requisitos del cliente.
     \item<2-> El jefe no es programador.
     \item<3-> Pruebas escasas o inexistentes.
     \item<4-> Dificultad para probar.
     \item<5-> Ausencia de documentación.
     \item<6-> Tarde y mal.
    \end{itemize}
   \end{frame}

   {
   \usebackgroundtemplate{\includegraphics[width=\paperwidth]{images/panico.jpg}}

   \begin{frame}
    \frametitle{Programación tradicional}
    \framesubtitle{Resultados}

    \begin{itemize}
     \item \textcolor{white}{Jefes \textbf{cabreados}.}
     \item \textcolor{white}{Clientes \textbf{descontentos}.}
     \item \textcolor{white}{Programadores \textbf{desmotivados}.}
    \end{itemize}
   \end{frame}
   }

  \subsection{Movimiento Ágil}


  \begin{frame}
   \frametitle{Rompiendo la baraja}
   \framesubtitle{Robert C. Martin dice:}

   \begin{columns}
    \begin{column}{7cm}
     \begin{center}
    ``Estamos \textbf{hartos} de escribir \textbf{mierda}.''\\[3pc]
    (``We are tired of writing crap'')
     \end{center}
    \end{column}
    \begin{column}{4cm}
      \includegraphics[width=4cm]{images/mentorsRobertMartin.jpg}
    \end{column}
   \end{columns}
  \end{frame}


  {
  \usebackgroundtemplate{\includegraphics[width=\paperwidth]{images/camino.jpg}}
   \begin{frame}
    \frametitle{Rompiendo la baraja}
    \framesubtitle{}

    \transsplitverticalout<1>

    \vspace{4cm}
    \textcolor{white}{
    Si siempre haces lo que siempre has hecho,\\
    nunca llegarás más lejos de donde siempre has llegado.\\
    (Juan Bernat)
    }
   \end{frame}
  }


  {
   \usebackgroundtemplate{\includegraphics[width=\paperwidth]{images/agilemanifestbg.jpg}}

   \begin{frame}
    \frametitle{Agile Manifest}
    \framesubtitle{Manifiesto Ágil}

    \transsplitverticalout<1>
    \transdissolve<2->


    \only<1>{
    Estamos descubriendo formas mejores de desarrollar\\
    software tanto por nuestra propia experiencia como\\
    ayudando a terceros. A través de este trabajo hemos\\
    aprendido a valorar:
    }
    \only<2>{
    \begin{itemize}
     \item \textbf{Individuos e interacciones} sobre procesos y herramientas
     \item \textbf{Software funcionando} sobre documentación extensiva
     \item \textbf{Colaboración con el cliente} sobre negociación contractual
     \item \textbf{Respuesta ante el cambio} sobre seguir un plan
    \end{itemize}

    Esto es, aunque valoramos los elementos de la derecha,
    valoramos más los de la izquierda.
    }
    \only<3>{
    \begin{columns}
     \begin{column}{4cm}
      Kent Beck\\
      Mike Beedle\\
      Arie van Bennekum\\
      Alistair Cockburn\\
      Ward Cunningham\\
      Martin Fowler
     \end{column}
     \begin{column}{3cm}
      James Grenning\\
      Jim Highsmith\\
      Andrew Hunt\\
      Ron Jeffries\\
      Jon Kern\\
      Brian Marick
     \end{column}
     \begin{column}{3cm}
      Robert C. Martin\\
      Steve Mellor\\
      Ken Schwaber\\
      Jeff Sutherland\\
      Dave Thomas
     \end{column}
    \end{columns}
    }
   \end{frame}

   \usebackgroundtemplate{}

  \begin{frame}
   \frametitle{Agile Manifest}
   \framesubtitle{Individuos e Interacción vs Procesos y Herramientas}

   \begin{columns}
    \begin{column}{4cm}
     \includegraphics[width=4cm]{images/herramientas.jpg}
    \end{column}
    \begin{column}{7cm}
     \begin{itemize}
      \item<1-> Manejar herramientas requiere conocimientos.
      \item<2-> Un buen proceso/herramienta no sirve si los individuos no lo usan, no saben usarlo o no quieren usarlo.
     \end{itemize}
    \end{column}
   \end{columns}
  \end{frame}


  \begin{frame}
   \frametitle{Agile Manifest}
   \framesubtitle{Software que Funciona vs Documentación Exhaustiva}

   \begin{columns}
    \begin{column}{4cm}
     \includegraphics[width=4cm]{images/documentos.jpg}
    \end{column}
    \begin{column}{7cm}
     \begin{itemize}
      \item<1-> Prototipado.
      \item<2-> Documentación como barricada entre departamentos.
      \item<3-> En ocasiones, la documentación es necesaria (temas legales, información histórica, etc.)
     \end{itemize}
    \end{column}
   \end{columns}
  \end{frame}

  \begin{frame}
   \frametitle{Agile Manifest}
   \framesubtitle{Colaboración con el Cliente vs Negociación Contractual}

   \begin{columns}
    \begin{column}{7cm}
     \begin{itemize}
      \item<1-> A menudo, el cliente no sabe lo que quiere o no sabe explicarlo.
      \item<2-> Habrá información sobreentendida.
      \item<3-> Los requisitos cambian.
     \end{itemize}
    \end{column}
    \begin{column}{4cm}
     \includegraphics[width=4cm]{images/contrato.jpg}
    \end{column}
   \end{columns}
  \end{frame}

  \begin{frame}
   \frametitle{Agile Manifest}
   \framesubtitle{Respuesta al Cambio vs Seguir el Plan}

   \begin{columns}
    \begin{column}{7cm}
     \begin{itemize}
      \item<1-> El modelo ágil surge de entornos inestables y cambiantes.
      \item<2-> Se prefiere adaptación frente a previsión.
     \end{itemize}
    \end{column}
    \begin{column}{4cm}
     \includegraphics[width=4cm]{images/plan.jpg}
    \end{column}
   \end{columns}

  \end{frame}

  \begin{frame}
   \frametitle{Agile Manifest}
   \framesubtitle{¿Qué es lo importante?}

   \transsplitverticalout<1>

   \begin{center}
    No nos equivoquemos:\\
    ¡¡ Lo realmente importante es \\[1pc]
    \textbf{\Large la satisfacción del cliente} !!
   \end{center}
  \end{frame}


  \begin{frame}
   \frametitle{Agile Manifest}
   \framesubtitle{Con un cliente satisfecho:}

   \transdissolve<1>

    \begin{columns}
     \begin{column}{4cm}
      \includegraphics[width=4cm]{images/ganador.jpg}
     \end{column}
     \begin{column}{6cm}
      \begin{itemize}
       \item El jefe estará \textbf{contento}.
       \item El cliente estará \textbf{contento}.
       \item Los desarrolladores se sentirán \textbf{realizados}.
      \end{itemize}
     \end{column}
    \end{columns}

  \end{frame}


  % \begin{frame}
  %  \frametitle{Relax}
  %  \framesubtitle{}

  %  (Nota mental: La programación tradicional es... como un globo)

  % \end{frame}

 \begin{frame}
  \frametitle{Artesanía de software}
  \framesubtitle{Carlos Blé}
  \begin{columns}
   \begin{column}{6cm}
    El verdadero ideal de la artesanía del software es el de ser más
    profesionales, hacer mejor el trabajo, ponerle toda la atención
    posible y mejorar constantemente.
   \end{column}
   \begin{column}{5cm}
    \includegraphics[width=5cm]{images/mentorsBle.jpg}
   \end{column}
  \end{columns}
 \end{frame}
