 \section{Ejercicios}

  \subsection{Números romanos}

  \begin{frame}
   \frametitle{Enunciado}
   \framesubtitle{}

   Dado un número decimal (número arábigo), obtener el número romano
   correspondiente.

   \begin{columns}
    \begin{column}{4cm}
     \begin{exampleblock}
      {Valores}
      {
      1 -> I\\
      5 -> V\\
      10 -> X\\
      50 -> L\\
      100 -> C\\
      500 -> D\\
      1000 -> M\\
      }
     \end{exampleblock}

    \end{column}
    \begin{column}{4cm}
     \begin{exampleblock}
      {Ejemplos}
      {
      16 -> XVI\\
      49 -> XLIX \\
      499 -> CDXCIX \\
      1998 -> MCMXCVIII\\
      }
     \end{exampleblock}
    \end{column}
   \end{columns}

  \end{frame}

  \begin{frame}
   \frametitle{Solución}
   \huge {Mi solución}
  \end{frame}

  \begin{frame}
   \frametitle{Solución}

   \begin{testfail}
    \lstinputlisting[language=Python]{exer1/test1.py}
   \end{testfail}

  \end{frame}

  \begin{frame}
   \frametitle{Solución}

   \begin{codepass}
    \lstinputlisting[language=Python]{exer1/code1.py}
   \end{codefail}

  \end{frame}

  \begin{frame}
   \frametitle{Solución}

   \begin{testfail}
    \lstinputlisting[language=Python]{exer1/test2.py}
   \end{testfail}

  \end{frame}

  \begin{frame}
   \frametitle{Solución}

   \begin{codepass}
    \lstinputlisting[language=Python]{exer1/code2.py}
   \end{codepass}

  \end{frame}

  \begin{frame}
   \frametitle{Solución}

   \begin{testfail}
    \lstinputlisting[language=Python]{exer1/test3.py}
   \end{testfail}

  \end{frame}

  \begin{frame}
   \frametitle{Solución}

   \begin{codepass}
    \lstinputlisting[language=Python]{exer1/code3.py}
   \end{codepass}

  \end{frame}


  \subsection{Cruces de Calatrava}

  \begin{frame}
   \frametitle{Enunciado}
   \framesubtitle{}

   Se tiran dos monedas. Si salen cruces, cada apostante recupera el
   dinero que había jugado y gana la misma cantidad que hubiera
   apostado.  Si salen las dos caras, vence la banca.  Y si sale cara y
   cruz, en paz, nadie gana, volviéndose a repetir el lanzamiento de las
   dos piezas.


  \end{frame}


  \begin{frame}
   \begin{exampleblock}
    {}
    {hola}
   \end{exampleblock}
  \end{frame}
