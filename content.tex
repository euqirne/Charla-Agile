\section{Estadísticas}

\begin{frame}
  \frametitle{Estadísticas}

  \begin{center}
    {\Huge ¿Cuántos trabajáis?}
  \end{center}
\end{frame}

\begin{frame}
  \frametitle{Estadísticas}

  \begin{center}
    {\Huge ¿Sabéis programar?}
  \end{center}
\end{frame}

\begin{frame}
  \frametitle{Estadísticas}

  \begin{center}
    {\Huge ¿Escribís pruebas para vuestros programas?}
  \end{center}
\end{frame}

\begin{frame}
  \frametitle{Estadísticas}

  \begin{center}
    {\Huge ¿Sabéis en qué se diferencia un \textbf{Mock} de un \textbf{Stub}?}
  \end{center}
\end{frame}

\begin{frame}
  \frametitle{Estadísticas}

  \begin{center}
    {\Huge ¿Utilizáis JUnit?}
  \end{center}
\end{frame}

\begin{frame}
  \frametitle{Estadísticas}

  \begin{center}
    {\Huge ¿Conocéis la técnica Pomodoro?}
  \end{center}
\end{frame}

\begin{frame}
  \frametitle{Estadísticas}

  \begin{center}
    {\Huge ¿Y qué es Scrum?}
  \end{center}
\end{frame}

\section{Motivación}

\subsection{Malas prácticas}

\begin{frame}
  \frametitle{Programación tradicional}
  \framesubtitle{Lo que siempre nos han enseñado}

  \begin{itemize}
  \item Especificar \textbf{los requisitos}.
  \item \textbf{Aceptar} los requisitos.
  \item \textbf{Planificación} de tiempos.
  \item \textbf{Desarrollar} la aplicación.
  \item \textbf{Probar} que todo funciona.
  \item \textbf{Documentar} lo que se ha hecho.
  \item \textbf{Entregar} al cliente.
  \end{itemize}
\end{frame}

\begin{frame}
  \frametitle{Programación tradicional}
  \framesubtitle{Lo que realmente querían decir}

  \begin{itemize}
  \item Hacer la carta a \textbf{los Reyes Magos}.
  \item \textbf{Dividir} el tiempo que digan los programadores entre 10.
  \item Dejar que los programadores \textbf{se diviertan}.
  \item \textbf{Fustigar} a los programadores por no tenerlo a tiempo.
  \item \textbf{Transmitir} a los programadores las quejas del cliente.
  \end{itemize}
\end{frame}


\begin{frame}
  \frametitle{Programación tradicional}
  \framesubtitle{Causas}

  \begin{itemize}
  \item Carta de los Reyes Magos != Requisitos del cliente.
  \item El jefe no es programador.
  \item Pruebas escasas o inexistentes.
  \item Dificultad para probar.
  \item Ausencia de documentación.
  \item Tarde y mal.
  \end{itemize}
\end{frame}


\begin{frame}
  \frametitle{Programación tradicional}
  \framesubtitle{Resultados}

  \begin{itemize}
  \item Jefes \textbf{cabreados}.
  \item Clientes \textbf{descontentos}.
  \item Programadores \textbf{desmotivados}.
  \end{itemize}
\end{frame}


\begin{frame}
  \frametitle{Programación tradicional}
  \framesubtitle{Robert C. Martin dice:}

  ``Estamos \textbf{hartos} de escribir \textbf{mierda}.''\\[3pc]
  (``We are tired of writing crap'')

\end{frame}


\begin{frame}
  \frametitle{Programación tradicional}
  \framesubtitle{}

  Si siempre haces lo que siempre has hecho,\\
  nunca llegarás más lejos de donde siempre has llegado.

\end{frame}


\subsection{Movimiento Ágil}

\begin{frame}
  \frametitle{Agile Manifest}
  \framesubtitle{Manifiesto Ágil}
Estamos descubriendo formas mejores de desarrollar
software tanto por nuestra propia experiencia como
ayudando a terceros. A través de este trabajo hemos
aprendido a valorar:

 \begin{itemize}
 \item \textbf{Individuos e interacciones} sobre procesos y herramientas
 \item \textbf{Software funcionando} sobre documentación extensiva
 \item \textbf{Colaboración con el cliente} sobre negociación contractual
 \item \textbf{Respuesta ante el cambio} sobre seguir un plan
  \end{itemize}

Esto es, aunque valoramos los elementos de la derecha,
valoramos más los de la izquierda.


\end{frame}


\begin{frame}
  \frametitle{Agile Manifest}
  \framesubtitle{Firmantes}
    \begin{columns}
    \begin{column}{4cm}
      Kent Beck\\
      Mike Beedle\\
      Arie van Bennekum\\
      Alistair Cockburn\\
      Ward Cunningham\\
      Martin Fowler
    \end{column}
    \begin{column}{3cm}
      James Grenning\\
      Jim Highsmith\\
      Andrew Hunt\\
      Ron Jeffries\\
      Jon Kern\\
      Brian Marick
    \end{column}
    \begin{column}{3cm}
      Robert C. Martin\\
      Steve Mellor\\
      Ken Schwaber\\
      Jeff Sutherland\\
      Dave Thomas
    \end{column}
  \end{columns}
\end{frame}



\begin{frame}
  \frametitle{Agile Manifest}
  \framesubtitle{Individuos e Interacción vs Procesos y Herramientas}

  \begin{itemize}
  \item Manejar herramientas requiere conocimientos.
  \item Un buen proceso/herramienta no sirve si los individuos no lo usan, no saben usarlo o no quieren usarlo.
  \end{itemize}

\end{frame}


\begin{frame}
  \frametitle{Agile Manifest}
  \framesubtitle{Software que Funciona vs Documentación Exhaustiva}

  \begin{itemize}
  \item Prototipado.
  \item Documentación como barricada entre departamentos.
  \item En ocasiones, la documentación es necesaria (temas legales, información histórica, etc.)
  \end{itemize}

\end{frame}

\begin{frame}
  \frametitle{Agile Manifest}
  \framesubtitle{Colaboración con el Cliente vs Negociación Contractual}

  \begin{itemize}
  \item A menudo, el cliente no sabe lo que quiere o no sabe explicarlo.
  \item Habrá información sobreentendida.
  \item Los requisitos cambian.
  \end{itemize}

\end{frame}

\begin{frame}
  \frametitle{Agile Manifest}
  \framesubtitle{Respuesta al Cambio vs Seguir el Plan}

  \begin{itemize}
  \item El modelo ágil surge de entornos inestables y cambiantes.
  \item Se prefiere adaptación frente a previsión.
  \end{itemize}

\end{frame}

\begin{frame}
  \frametitle{Agile Manifest}
  \framesubtitle{¿Qué es lo importante?}

  \begin{center}
    No nos equivoquemos:\\
    ¡¡ Lo realmente importante es \\[1pc]
    \textbf{\Large la satisfacción del cliente} !!
  \end{center}
\end{frame}



\begin{frame}
  \frametitle{Agile Manifest}
  \framesubtitle{Con un cliente satisfecho:}

  \begin{itemize}
  \item El jefe estará \textbf{contento}.
  \item El cliente estará \textbf{contento}.
  \item Los desarrolladores se sentirán \textbf{realizados}.
  \end{itemize}
\end{frame}


\begin{frame}
  \frametitle{Relax}
  \framesubtitle{}

  (Nota mental: aquí es cuando enseñas las pelotas)

\end{frame}


\section{Técnicas}
Sprints
\subsection{Palabrotas}

\begin{frame}
  \frametitle{Scrum, Kanban, Scrumban}
  \framesubtitle{Qué son}

  \begin{itemize}
  \item Son marcos de trabajo (\textit{frameworks}).
  \item 4 tonterías que funcionan.
  \item Evitan la mentira.
  \end{itemize}
\end{frame}

\begin{frame}
  \frametitle{Scrum, Kanban, Scrumban}
  \framesubtitle{Scrum en 6 palabras}

  \begin{itemize}
  \item División en \textbf{Sprints}.
  \item Reunión de \textbf{planificación} de Sprint.
  \item \textbf{Scrum} diario.
  \item Reunión de \textbf{seguimiento} de Sprint opcional.
  \item Reunión de \textbf{evaluación} de Sprint.
  \item \textbf{Panel}: Pendiente, Asignado, Terminado. Burndown.
  \end{itemize}
\end{frame}

\begin{frame}
  \frametitle{Scrum, Kanban, Scrumban}
  \framesubtitle{Scrum en 6 palabras}

  \begin{itemize}
  \item División en \textbf{Sprints}.
  \item Reunión de \textbf{planificación} de Sprint.
  \item \textbf{Scrum} diario.
  \item Reunión de \textbf{seguimiento} de Sprint opcional.
  \item Reunión de \textbf{evaluación} de Sprint.
  \item \textbf{Panel}: Pendiente, Asignado, Terminado. Burndown.
  \end{itemize}
\end{frame}

\begin{frame}
  \frametitle{Scrum, Kanban, Scrumban}
  \framesubtitle{Test Driven Development}

  FIXME: Foto tablón scrum REAL

\end{frame}

\begin{frame}
  \frametitle{Scrum, Kanban, Scrumban}
  \framesubtitle{Test Driven Development}

  FIXME: Foto whiteboard de kunagi

\end{frame}

\begin{frame}
  \frametitle{Scrum, Kanban, Scrumban}
  \framesubtitle{Test Driven Development}

  FIXME: Foto dashboard de kunagi

\end{frame}

\begin{frame}
  \frametitle{Scrum, Kanban, Scrumban}
  \framesubtitle{Test Driven Development}

  FIXME: Foto Burndown Agilefant

\end{frame}

\begin{frame}
  \frametitle{Scrum, Kanban, Scrumban}
  \framesubtitle{Agilefant}

  FIXME: Foto Backlog Agilefant

\end{frame}

\subsection{TDD}

\begin{frame}
  \frametitle{Test Driven Development}
  \framesubtitle{Qué significa}

  Pasos:
  \begin{itemize}
  \item Se escribe el test.
  \item Se escribe el código mínimo que pasa el test.
  \item Refactorizar.
  \end{itemize}
\end{frame}


\begin{frame}
  \frametitle{Código limpio}
  \framesubtitle{Importancia}

  \begin{itemize}
  \item El código se escribe 1 vez, pero se lee muchas.
  \item Las pruebas pueden ser la mejor documentación.
  \item Tantos las pruebas como el código nos cuentan una historia.
  \end{itemize}
\end{frame}


\begin{frame}
  \frametitle{Código limpio}
  \framesubtitle{Importancia de las pruebas}

  Escribir pruebas \ldots
  \begin{itemize}
  \item \ldots nos ayudará a escribir mejores pruebas.
  \item \ldots nos obligará a escribir código fácil de probar.
  \item \ldots mejorará la comprensión y usabilidad de nuestro código.
  \end{itemize}
\end{frame}

\begin{frame}
  \frametitle{Código limpio}
  \framesubtitle{Importancia de las pruebas}

  Escribir pruebas \ldots
  \begin{itemize}
  \item \ldots nos ayudará a escribir mejores pruebas.
  \item \ldots nos obligará a escribir código fácil de probar.
  \item \ldots mejorará la comprensión de nuestro código.
  \end{itemize}
\end{frame}
